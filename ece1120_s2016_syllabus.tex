\documentclass{article}
\title{ECE 1125 -- C Programming for Electrical and Computer Engineering \\ Spring 2016 \\ The George Washington University}
\author{Lecturer: Ahsen J. Uppal (\texttt{auppal@gwu.edu}) \\
Teaching Assistant: Pradeep Kumar (\texttt{pradeepk@email.gwu.edu})}

\date{}
\begin{document}

\maketitle

\section{Introduction}
This course introduces undergraduate students to computer programming
and solving engineering problems using the C programming
language. Students will learn to analyze, decompose, and translate
real-world problems into computational problems. Special emphasis will
be on writing clear well-structured code.

Topics include basic programming concepts, including algorithmic
thinking and structured programming, control flow, data types,
pointers, functions, algorithms, I/Os, threads, and performance
evaluation and optimization. Advanced topics include concurrency and
multicore programming using threads, processes as well as parallel C
programming paradigms, and controlling hardware devices and fine
control via interfacing with assembly language.
\\

\noindent
\textbf{Students are required to honor the GWU Code of Academic Integrity when completing all assignments, projects, and examinations.}

\subsection{Textbooks}
Stephen G. Kochan, \textit{Programming in C} (4th Edition). \\
Brian W. Kernighan and Dennis M. Ritchie, \textit{The C Programming Language} (2nd Edition).

\subsection{Office Hours}
The instructor's office hours are Tuesdays and Thursdays 9:30am - 10:30am in SEH 5750, and by appointment.


\section{Grading}
\begin{center}
\begin{tabular}{|l|l|}
\hline
Participation (Including Lab Attendance) & 10\% \\
Homeworks and Lab Assignments & 25\% \\
Projects (One or Two Major Projects) & 20\% \\
Midterm   &     20\% \\
Final Exam &    25\% \\

\hline
\end{tabular}
\end{center}

\subsection{Policies}
\begin{itemize}
\item Participation: Each student must be willing to participate fully
  in the class. It is necessary, but not sufficient to just show up to class.
\item Assignments: You will be assigned homework and lab assignments to complete under guidance from the teaching assistant.
\item Projects: There will be one or two major projects during the
  semester, each requiring a significant amount of time to finish. You
  are strongly encouraged to talk to the instructors often and seek
  help as soon as possible.
\item \textbf{Late Penalty: Note that there is a 20\% per day penalty for each day an assignment is late.}
\end{itemize}

\subsection{Course Contents}
\subsubsection{Introduction to Computer Systems}
\begin{itemize}
\item Engineering problems as computational problems
\item Overview of computer systems
\item Software design
\end{itemize}

\subsubsection{Introduction to C}
\begin{itemize}
\item Code build process (editing, compiling, linking, executing)
\item Elements of a C program; preprocessor directives; statements and expressions; functions;
coding formatting style
\item Simple data types; constants and variables; conversion between different data types;
binary arithmetic representations
\item The integrated development environment (IDE)
\end{itemize}

\subsubsection{Program Flow Control}
\begin{itemize}
\item Conditions; relational operators; logical operators; precedence rules; selection structures
\item Repetition and loop statements; while statements; for statements; increment and
decrement operators; loop termination; nested loops; do-while statements
\item Debugging
\end{itemize}

\subsubsection{Modular Programming}
\begin{itemize}
\item User functions; library functions; function declaration and definition; function calls; pass
by value; scope rules; programs with multiple functions
\item Pointers and addresses; pass by reference; pointer arithmetic
\item File input/output
\end{itemize}

\subsubsection{Simple data structures}
\begin{itemize}
\item Arrays; declaration and initialization; multi-dimensional arrays; searching and sorting
arrays; pointers and arrays
\item String arrays; string library functions; substrings; concatenation; strings vs. characters
\item Engineering applications; matrix algebra; numerical integration and differentiation;
quadratic equations
\item Recursion
\item Structures; structures and functions; arrays of structures; dynamic data structures
\end{itemize}

\makeatletter
\def\blfootnote{\gdef\@thefnmark{}\@footnotetext}
\makeatother

\blfootnote{Last modified \today.}


\end{document}
